\chapter{Introduction}
\section{Kotlin and Scala.js}

\paragraph{} JavaScript (JS for short) is a programming language that provides developers with a 
way to interact with users by building dynamic web applications and websites. Through the use of 
JavaScript runtimes, such as Node.js~\cite{nodejs}, it is no longer required that applications be 
executed only on the client side. It can therefore also be used to build a large range of 
server applications. Many frameworks and libraries exist to provide useful functionalities to the 
language and ease the development of said applications.

\paragraph{} JavaScript is a weakly typed  language and provides a large freedom on how to organize
one's code. However, depending on the situation, a strongly typed language might be more suitable for development. This is where Kotlin JS and Scala.js come into play.

\paragraph{Kotlin JS} Kotlin~\cite{kotlin} is a statically typed programming language providing 
many features through its large standard library. It is able to run on the JVM and is officially 
supported for Android since May 2017~\cite{android_kotlin_support}. In order to expand their reach, 
the Kotlin team has built a JavaScript compiler to translate the Kotlin standard library 
and the language capabilities into a JS application. It however lacks some optimizations and 
produces a rather large amount of JS code.

\paragraph{Scala.js} Scala.js~\cite{scalajs} is another typed programming language. Being 
implemented as a Scala compiler plugin, it benefits from most of its features with a few 
differences due to the JavaScript semantics (more details about this can be found in the 
\enquote{Hands on Scala.js} tutorial~\cite{scalajs_handson}).

\paragraph{} Both languages provide specific language features in order to allow developers to interact with 
pre-existing JavaScript objects and thereby provide JavaScript interoperability. The main 
difference between Kotlin JS and Scala.js is the number of optimizations performed on the code before 
outputting JavaScript. The sizes of the final JS files are also very different. 

\paragraph{} Scala.js (mainly thanks to its linker) enhances the performance by performing whole 
program optimizations and greatly reduces the size of the generated code. All details about the 
Scala.js compilation pipeline (and how it optimizes JavaScript code) can be found under section 
\enquote{The compilation pipeline} of the \enquote{Hands on Scala.js}~\cite{scalajs_handson} 
tutorial. Kotlin, since its version 1.1.4, provides Dead Code Elimination through an additional 
Gradle plugin which can be found in the Kotlin documentation~\cite{kotlin_dce}.


\paragraph{} As a summary, combining both the Kotlin language flexibility and the Scala.js 
optimizing capabilities can result in a great improvement for JavaScript applications development 
both in terms of executable size and performance.

\section{Motivations}
\paragraph{} I have always been strongly interested in web related technologies and the 
opportunity to work at the core of one of them is really an enriching experience. The Kotlin 
compiler is a very large code base and having to search trough its numerous source files in order 
to understand the logic behind the language is challenging and at the same time captivating.

\paragraph{} Besides the performance aspect behind the compilation pipeline (and of course the 
language syntax), one of the most important things for a language to be adopted is the tooling 
available and how easy it is to setup a new project and compile one's code.
    
\paragraph{} The main goal of this project was thus to provide developers with basic 
tooling and support for the Kotlin language. This will be achieved by continuing prior work to 
compile the Kotlin standard library to the Scala.js IR and by writing a simple Gradle~\cite{gradle} 
plugin.

\section{Project structure and architecture}

\paragraph{} The work on a Kotlin to Scala.js compiler was started in February 2017 by Lionel 
Fleury and Guillaume Tournigand. Their work resulted in a structured code base which allowed simple 
Kotlin programs to compile (for more details on their work, please refer to their semester project 
report~\cite{kotlin_scalajs_report}). A brief review of the working features of their project is 
available under section \ref{initial_status}.

\subsection*{Main source code}

\paragraph{} The main code of the project is stored under the \scalainline{ch.epfl.k2sjsir} 
(root) package, organized in three subpackages: \scalainline{lower}, \scalainline{translate} and 
\scalainline{utils}.

\paragraph{} More details about this design and the underlying Kotlin AST (PSI, Descriptors and 
Contexts) can be found under section \enquote{Kotlin} of Guillaume's and Lionel's 
report~\cite{kotlin_scalajs_report}.

\paragraph{Package root} A few files are located at the root of the \scalainline{ch.epfl.k2sjsir} 
package and they are the entry point into the Kotlin compiler pipeline. For reference, the 
\scalainline{translate} package is first called via the \scalainline{GenClass} class from the 
\scalainline{PackageDeclarationTranslator} file.
      
\paragraph{Package \scalainline{lower}} This package contains all the logic related to class 
lowering which is used to retrieve inner classes and objects (besides anonymous objects, see 
section \ref{anonymous_objects}) from the Kotlin AST.

\paragraph{Package \scalainline{translate}} This is where the main translation logic is kept. 
It is organized in a generator pattern and thus every class in this package must provide a way to 
retrieve an SJSIR object (generally through a \scalainline{tree} method) from a \ktinline{KtElement}. 
For example if you are provided with a \scalainline{KtExpression} object, you can translate it with:

\begin{minted}[linenos,frame=leftline]{scala}
  val e: KtExpression = /* Retrieve a KtExpression */
  GenExpr(e).tree // This will return a Scala.js IR node
\end{minted}


\paragraph{Package \scalainline{utils}} This last package contains a variety of utility 
methods such as implicit classes which provide an interface to encode names, retrieve types or 
generate recurring SJSIR objects, etc.

\subsection*{Tests}
\paragraph{} In order to verify that no code change breaks the translation of other elements, a 
test suite exists under the \scalainline{test} directory. This directory is organized in two main 
folders which are \scalainline{resources} and \scalainline{scala}.

\paragraph{Folder \scalainline{resources}} There are mainly two directories in this folder; the 
first one, \scalainline{lib}, contains the Scala.js library jar file. These are used when launching 
the Scala.js linker to compile the SJSIR files output by the compiler. The second folder is the 
\scalainline{src} folder which contains all Kotlin sources to test the good behavior of the 
Kotlin to Scala.js compiler.

\paragraph{Folder \scalainline{scala}} It contains the various ScalaTest~\cite{scala_test} files 
which define the unit tests. Each test file was written to test a specific language feature, 
please refer to section \ref{features} for details about which Kotlin feature these test files are 
related to. The corresponding source code can be found in the GitHub 
repository~\cite{kotlin_scalajs_v2}.

\subsection*{General design of the project}

\paragraph{Entry point} The \scalainline{K2SJSIRCompiler} class is used in order to inject the 
Scala.js IR translation logic into the Kotlin compiler. It is where the parsing of the 
command line arguments happens, through the use of the \scalainline{K2SJSIRCompilerArguments} class. The 
resolution of source file paths is then handed to the \scalainline{K2SJSIRTranslator} which in turn 
will delegate the work to the \scalainline{Translation} class.

\paragraph{} Finally, this class will hand the task of generating SJSIR nodes to the
\scalainline{PackageDeclarationTranslator} class. It then makes use of the content of the
\scalainline{translate} package in order to generate the final SJSIR files.

\paragraph{} Since this code is (mostly) a Scala translation of the original  
\scalainline{K2JSCompiler} class, all phases (parsing, type checking, etc.) of the original 
compiler are still there.

%% Compiler pipeline illustration

\label{translation}
\paragraph{Translation} As explained above, the translation happens mainly inside the 
\scalainline{PackageDeclarationTranslator} class. It will extract the various declarations using 
lowering and generate the corresponding SJSIR files. The declarations can be either classes and
objects or top-level functions and properties.

\paragraph{} The reasoning behind this design is explained in more details in the \enquote{Hello 
World} section of Guillaume and Lionel's report~\cite{kotlin_scalajs_report}.

\begin{comment}
  - talk about kotlin representation ?
  - project organization (folders) --> README ?
  - general design of the compiler
  - supported kotlin + scalajs versions
  Annexes
  - brief explanation of the various utils available
  - definition of dispatch receiver, extension receiver
  - kotlin standard library compilation (which files and how)
\end{comment}