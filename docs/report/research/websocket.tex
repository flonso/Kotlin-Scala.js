\section{WebSockets}

\subsection{History}

\paragraph{} This last technology is the latest one. The first public draft 
regarding the WebSocket API was published in April 2009 and was followed by a 
few others between this date and September 2012 \cite{websocket-history}.

WebSockets were designed to palliate the miss of a full-duplex, bi-directional 
transmission channel for client-server communications.

\paragraph{} \cite{guide-html5-websocket} As opposed to the Ajax and Comet 
techniques, WebSocket is a web protocol based on the existing TCP/IP and HTTP 
protocols which are at the root of almost every client-to-server web 
interactions. The WebSocket protocol inherits mainly the asynchronous feature 
of TCP and the text support of HTTP.


%\subsection{How it works}

\subsection{Chat application}

\paragraph{} The WebSocket protocol is supported as for today by a lot of 
different programming languages and libraries. That is why I will present 
the standard WebSocket handshake protocol and will describe a short example 
through the use of Javascript native WebSocket object.

\paragraph{WebSocket Handshake} A WebSocket connection can only be started 
through an HTTP request. The reason why is that this ensures full compatibility 
with pre-WebSocket web servers.

The following Javascript will initiate the handshake client-side :

\begin{minted}[linenos, frame=leftline]{javascript}
    var socket = new WebSocket(
                "ws://echo.websocket.org/?encoding=text"
            );
\end{minted}

\paragraph{} Once the WebSocket handshake is complete, the HTTP protocol 
connection is shut down and is replaced by the WebSocket one on the same TCP/IP 
channel.

Figures \ref{fig:ws-handshake-request} and 
\ref{fig:ws-handshake-response} illustrate a standard handshake request and 
response to the websocket.org echo\footnote{An echo server is a server sending 
back the original message} server.

\paragraph{} The programmer will then be able to listen to new incoming 
messages on the connection and react accordingly client-side through the native 
Javascript WebSocket API.

\begin{figure}[h]
    \begin{minted}[frame=leftline, linenos]{text}
    GET ws://echo.websocket.org/?encoding=text HTTP/1.1
    Origin: http://websocket.org
    Cookie: __utma=99as
    Connection: Upgrade
    Host: echo.websocket.org
    Sec-WebSocket-Key: uRovscZjNol/umbTt5uKmw==
    Upgrade: websocket
    Sec-WebSocket-Version: 13
    \end{minted}
    \caption{WebSocket request sent by client \cite{websockets.org}}
    \label{fig:ws-handshake-request}
\end{figure}


\begin{figure}[h]
    \begin{minted}[frame=leftline, linenos]{text}
    HTTP/1.1 101 WebSocket Protocol Handshake
    Date: Fri, 10 Feb 2012 17:38:18 GMT
    Connection: Upgrade
    Server: Kaazing Gateway
    Upgrade: WebSocket
    Access-Control-Allow-Origin: http://websocket.org
    Access-Control-Allow-Credentials: true
    Sec-WebSocket-Accept: rLHCkw/SKsO9GAH/ZSFhBATDKrU=
    Access-Control-Allow-Headers: content-type
    \end{minted}
    \caption{WebSocket response sent by server \cite{websockets.org}}
    \label{fig:ws-handshake-response}
\end{figure}

\paragraph{} A concrete chat protocol can be quickly derived from this :

\begin{itemize}
 \item A user displays a page in his browser and loads all referenced 
Javascripts.
 \item One of the scripts initiates a WebSocket connection to a server and 
attach different listeners to it.
 \item When the message listener detects new incoming data, it displays them in 
the web page.
 \item The user can write text in a text field and click a send button which 
will trigger Javascript whose task will be to transmit the message through the 
opened WebSocket connection.
\end{itemize}

\subsection{Pros and cons}

\paragraph{} Serious advantages of the WebSocket protocol are its truly 
full-duplex and bi-directional communication possibilities. Thanks to this the 
communication delays are considerably reduced.

\paragraph{} Another advantage is its native support in most recent web 
browsers through the Javascript language and also the countless number of 
libraries providing support for it. A lot of them also provide different 
automated fallback methods in case of a non-websocket supported environment 
(however, they often require a server counterpart).

\paragraph{} In comparison to the previously explained techniques, Websockets 
also 
benefit of a smaller amount of data to be transferred on each request. Cookies 
and all request headers are only transmitted once on the handshake.

\paragraph{} Since WebSockets are wrapped by a TCP channel, the server will know 
quickly when the client disconnects without the need of using timeouts to 
detect these cases.

\paragraph{} One disadvantage at this point is that this technology is fairly 
recent and its support on web servers often requires the installation of 
additional libraries and packages.

\paragraph{} Another downside of this protocol is the following : depending on 
a client's configuration and on what network the client is connected to, the 
connection will be handled differently, for instance, when surfing on mobile 
devices through the mobile network. As a matter of fact, some 
ISPs\footnote{Internet Service Provider} have configured proxy servers which 
will act as a bridge between the mobile device and the server. As a consequence 
of this the connection is expiring sooner than expected after some time of 
inactivity on the 
client side.