\section{Ajax}

\subsection{History}

\paragraph{Iframe Tag} Different techniques \cite{ajax-wikipedia} appeared to 
palliate the problem of refreshing only part of a file, one of which was the 
iframe tag in 1996 : this HTML tag, introduced by the famous Microsoft's 
Internet Explorer, allowed to fetch only data from a server by embedding 
other HTML documents in an existing web page.

\paragraph{XMLHttpRequest Object} A few years after the introduction of this 
tag, a Microsoft team implemented the first XMLHTTP component which will later 
be adopted by the well known Mozilla, Safari and Opera browsers as well as by 
others as the XMLHttpRequest API often abbreviated XHR.

This API works on top of the HTTP protocol and allows programmers to 
send HTTP requests after the page has been loaded in the browser and to handle 
the server response in the background. This could, for example, be used to add 
later some retrieved contents to the page by manipulating the HTML DOM 
elements.

\paragraph{} The Ajax terminology was first publicly stated in 2005 by Jesse 
James Garrett in an article for Adaptive Path\cite{ajax-adaptivepath}, a 
web consulting company. The Ajax \cite{ajax-w3schools} acronym stands for 
Asynchronous Javascript and XML\footnote{Extensible Markup Language}.

\paragraph{} The Ajax technique was built with the goal of creating fast and 
dynamic web pages in mind : with Ajax it is possible to update only 
some parts of a web page without reloading a whole page from the server and 
this by exchanging small amounts of data with the server in background tasks.

\subsection{Chat application}

\begin{figure}[h]
 \begin{center}
    \includegraphics[width=0.5\textwidth]{imgs/ajax-poll.jpg}
 \end{center}
 \caption{HTTP polling with Ajax \cite{ibm-reverse-ajax}}
\end{figure}

\paragraph{} In order to satisfy the aforementioned requirements, a developed 
chat application using the Ajax technique would have to rely on a 
protocol similar to the one described hereunder :


\begin{itemize}
 \item A user displays a page in the browser loading all required Javascript 
dependencies.
 \item The application will create an XMLHttpRequest object and initialize the 
discussion system, for instance by querying the server and retrieving a list of 
the last messages.
 \item The user's browser will then have to poll the server regularly in order 
to retrieve the newest messages from the server.
\end{itemize}


\subsection{Pros and cons}

\paragraph{} The main advantage of the Ajax technique is that it is very easy 
to put it in use since there are countless libraries which provide an 
easy-to-use and cross-browser compatible interface.

\paragraph{Example} The code sample which can be seen on figure 
\ref{ajax-jquery} will send a user's name and message to the server (more 
precisely the data will be treated by a php script called receive-msg.php) and 
the user will get a confirmation popup if successful. In case of a 404 
error\footnote{Page not found error code} the user would also be warned.

\begin{figure}[h]
    \begin{minted}[linenos, frame=leftline]{javascript}
    $.ajax(
        type: "POST",
        url: "receive-msg.php",
        data: { name: "John Doe", msg: "Hello World !"},
        success : function(data, status, xhrObj) {
            alert("Message sent successfully !");
        },
        statusCode : {
            404: function() {
                alert("Page not found");
            }
        }
    );
    \end{minted}
    \caption{An Ajax example in JQuery}
    \label{ajax-jquery}
\end{figure}

\paragraph{} The main disadvantage of this technique is that the client will have 
to poll the server on a regular basis in order to provide some real time 
updates. Furthermore the polling interval will have to be relatively short so 
that the user won't notice any inconvenient delays.

\paragraph{} The server will have a hard time to make any difference between 
a definitive disconnection of the client and a disconnection interval between 
two polls. This would need the use of server-side timeouts: on disconnection, 
if 
the user does not reconnect within a fixed delay it is removed from the online 
users' list.

\paragraph{} Another major problem of such a technique is that as the number 
of users grows, the number of concurrent requests to the web server grows as 
well, thus inducing a noticeable increase in the need of physical resources 
required to handle such a traffic load. Such a behaviour leads this type of 
application to be non scalable.