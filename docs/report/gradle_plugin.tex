
\chapter{Gradle tooling}
\section{About Gradle}
\paragraph{} Gradle~\cite{gradle} is a build tool allowing developers to define the build logic of 
their project programmatically. This logic can be separated in various tasks performing independent 
operations. The defined tasks can later be linked among them by defining dependencies in a build 
script.

\paragraph{} The build scripts defining this logic are written in a special DSL (Domain Specific 
Language) format and since not too long ago in a Kotlin-based DSL. This is probably not by chance 
since this tool is used by a majority of Kotlin developers (mostly Android) and the Kotlin team 
itself in the compiler source code. 

\paragraph{} Besides writing build scripts, developers can write custom plugins 
~\cite{gradle_plugins} which can be applied to existing projects in order to provide reusable logic.

\section{Compiler plugin}

\paragraph{} Kotlin JS provides its own Gradle plugin called \ktinline{kotlin2js} which can be 
applied in a Gradle project by defining a few dependencies and applying the plugin itself with 
\ktinline{apply plugin: 'kotlin2js'}.

\paragraph{} The goal for the Kotlin to Scala.js compiler is to provide a similar way of applying our compiler with minor differences with the original one. The source code of the plugin can be found on GitHub~\cite{kotlin_scalajs_gradle_plugin} along with a simple example project.

\subsection{Plugin architecture}

\paragraph{} The plugin is built thanks to the Java Gradle Development Plugin~\cite{gradle_plugins} 
developped by the Gradle team. It provides some useful utilities in order to declare a Gradle 
plugin and use it in other projects. The original \ktinline{kotlin2js} plugin is also used in order 
to provide support for Kotlin SourceSets and allow building a project with the two compilers (see 
under "Available tasks").

\paragraph{} Because there are limited examples and documentation about plugin development, all 
sources are written in Groovy~\cite{groovy}. The base of the plugin is defined inside the 
\ktinline{K2SJSCompilerPlugin} file, this is where the compilation task is created and where it's 
made possible to call it using \ktinline{gradle build} (see section \ref{plugin_usage} for more 
details).

\paragraph{} The compilation of Kotlin sources is achieved by reproducing the compilation 
pipeline described in section \ref{translation}. In other words, Gradle will invoke the Kotlin to 
Scala.js compiler with the referenced sources and then hand the generated \scalainline{.sjsir} 
files to the Scala.js linker for further treatment.

\subsection{Plugin usage} \label{plugin_usage}

\subsubsection{Applying the plugin}

\paragraph{} The basic content of the \ktinline{build.gradle} file of a Kotlin to Scala.js project 
is presented in figure \ref{plugin_build_file}.

\paragraph{} Instructions on lines 14 and 15 will add the two dependencies required for the plugin 
to work, mainly the \ktinline{kotlin-gradle-plugin} and the \ktinline{kotlin2sjs} plugin itself.

\paragraph{} In order to load the Scala.js library at compile time, it is necessary 
that the \ktinline{dependencies} block contains a reference to the right library version 
(see line 27). Since this is an external dependency, the \ktinline{repositories} block must be 
defined as well.

\paragraph{} If you wish to compile your code using the original Kotlin compiler, you need to add the corresponding dependencies to the build script. Please refer to the \enquote{Avaiable tasks} section below, to learn more on how to run the Kotlin compiler directly.

\begin{figure}[h!]
\begin{minted}[linenos, frame=leftline]{kotlin}
group "ch.epfl.k2sjsir.example"

version "1.0-SNAPSHOT"

buildscript {
  ext.kotlin_version = "1.1.61"
    
  repositories {
    mavenCentral()
    mavenLocal()
  }

  dependencies {
    classpath "org.jetbrains.kotlin:kotlin-gradle-plugin:$kotlin_version"
    classpath "ch.epfl.k2sjs:kotlin2sjs:0.1-SNAPSHOT"
  }
}

apply plugin: "kotlin2sjs"


repositories {
  mavenCentral()
}

dependencies {  
  compile "org.scala-js:scalajs-library_2.12:1.0.0-M2"
  // If you want to make use of the original compiler as well
  //compile "org.jetbrains.kotlin:kotlin-stdlib-js:$kotlin_version"
}
\end{minted}
  \caption{A typical build.gradle file for a Kotlin to Scala.js project}
  \label{plugin_build_file}
\end{figure}



\subsubsection{Available tasks}

\paragraph{} This section lists the main available tasks defined by the plugin. A full 
list of these tasks and the ones generated by Gradle can be obtained by running \ktinline{gradle 
tasks} in the project root folder.

\paragraph{build} This is the standard Gradle build task which builds the sources of project. It 
depends on the \ktinline{k2sjs} task defined below.

\paragraph{build-original} This task provides an easy way to call the original Kotlin compiler in 
case one needs to compile a project with both compilers (for benchmarking purposes for instance).

\paragraph{clean} This is the standard Gradle clean task. It will erase the content of the 
\ktinline{build/} folder generated by the \ktinline{build} command.

\paragraph{k2sjs} This is the main compilation task defining the compilation steps detailed 
previously.

\subsubsection{Available options} \label{plugin_options}

\paragraph{} Various options are available in order to adapt the \ktinline{kotlin2sjs} plugin to 
one's needs. They must be declared and used inside the \ktinline{k2sjs} configuration block. An 
example configuration is provided in figure \ref{plugin_example_config}.


\begin{figure}[h]
\begin{minted}[linenos, frame=leftline]{kotlin}
  // Inside the build.gradle file
  k2sjs {
    kotlinHome = "/usr/share/kotlin"
    dstFile = "./web/js/mycode.js"
    compilerOptions = "-verbose"
    optimize = "fullOpt"
  }
\end{minted}
  \caption{An example configuration for the \ktinline{kotlin2sjs} plugin}
  \label{plugin_example_config}
\end{figure}

\paragraph{kotlinHome} Allows to specify the path to the local Kotlin installation directory. This 
option defaults either to the content of the environment variable 
\ktinline{\$KOTLIN_HOME} or to \ktinline{/usr/share/kotlin}. 

\paragraph{dstFile} This option contains the path and name of the .js file output by the linker. 
This must match the pattern \ktinline{path/to/output.js}. This option defaults to 
\ktinline{path/to/projectname/build/projectname.js}.

\paragraph{optimize} This option defines the optimization level desired for the output JavaScript 
code. It can contain one of the three following values : "noOpt", "fastOpt" or "fullOpt". Its 
default value is "fastOpt".

\paragraph{compilerOptions} Allows to specify custom compiler arguments. Those arguments must be 
contained in a String and be separated by spaces. By default it is empty.

\paragraph{linkerOptions} Just like the \ktinline{compilerOptions}, this field allows to pass 
custom command line arguments to the Scala.js linker. They must be separated by spaces as well. By 
default it is empty. Note that the option \enquote{-c} to check the consistency of the SJSIR files 
is always enabled.


% tasks available
% config options