\chapter{Initial Setup}

\paragraph{} As the last phase of this project was to integrate my work within 
uJoin's source code, the installation process is written for the project to 
work within it. I will later explain the dependencies required for both the 
client and the server applications.

\paragraph{} In the following sections \$LARAVEL\_ROOT, \$MOBILE\_ROOT and 
\$CHAT\_ROOT will refer to the uJoin's API engine installation directory, to 
the AngularJS client application directory and to this report source code 
directory respectively.

\section{Installing the server}

\subsection{Installing Ratchet}

\paragraph{} Installing Ratchet is very straightforward as it is installed 
automatically by Composer. Simply execute the following commands in the command 
line.

\begin{minted}{bash}
$ cd $LARAVEL_ROOT
$ composer require cboden/Ratchet:0.3.*
\end{minted}

\subsection{Installing missing dependencies}

\paragraph{} In order to install the chat server, you will need first to place 
the sources in a workbench directory located in \$LARAVEL\_ROOT :


\begin{minted}{bash}
$ mkdir $LARAVEL_ROOT/workbench/
$ cp -r $CHAT_ROOT/chatplugin $LARAVEL_ROOT/workbench/
\end{minted}

\paragraph{} After having copied the source code you will need to update the 
composer autoload file :

\begin{minted}{bash}
$ cd $LARAVEL_ROOT/workbench/chatplugin/chat
$ composer update
\end{minted}

\paragraph{} You need now to register the chat plugin, to open the \\
\$LARAVEL\_ROOT/app/config/app.php and to add the following line to the 
providers 
array entry :

\begin{minted}[linenos, frame=leftline]{php}
<?php
'providers' => array(
    ...
    'ChatPlugin\Chat\ChatServiceProvider',
    ...
),
?>
\end{minted}

\paragraph{} In order to verify your installation until this point, please run 
the command below. You should see the "Running for workbench 
[chatplugin/chat]..." message, meaning that the plugin was successfully 
registered.

\begin{minted}{bash}
$ cd $LARAVEL_ROOT
$ php artisan dump-autoload
\end{minted}

\subsection{Running migrations}

\paragraph{} In order for the chat server to handle connections correctly you 
will also need to run the package migrations. You can do this by running the 
following command :

\begin{minted}{bash}
 $ cd $LARAVEL_ROOT
 $ php artisan migrate --bench="chatplugin/chat"
\end{minted}

\subsection{Filling up missing source code}

\subsubsection{Ujoiner model}

\paragraph{} The chat server needs to be able to retrieve conversations related 
to a user. In order to do this, you need to add the following relation to the 
Ujoiner model file located in the \$LARAVEL\_ROOT/app/models/ directory.


\begin{minted}[linenos, frame=leftline]{php}
<?php
    public function conversations() {
        return $this->belongsToMany(
            'ChatPlugin\Chat\Models\ChatConversation', 
            'chat_conversation_ujoiner',
            'uJoinerID',
            'conversationID'
        );
    }
?>
\end{minted}

\paragraph{Note:} If you are not using uJoin's source code, it is possible to 
provide a different user model through the plugin's config.php file located in 
src/config. Please make sure that the model includes attributes id (32 bits, 
char), username, firstName, lastName and imageURI \footnote{relative path to 
the user's image location on the server, you may have to change the 
chat.conversation.html template in \$MOBILE\_ROUTE/www\_src/app/views/chat} 
otherwise the server won't work.

\subsubsection{OAuth2Controller}

\paragraph{} In order to be able to handle authentication without stopping, the 
OAuth2-Controller will need to be modified a little. Please update the 
verifyUserAccess method by first adding an optional boolean parameter like this:

\begin{minted}[linenos, frame=leftline]{php}
 <?php
 public function verifyUserAccess($fromChatServer = false)
 ?>
\end{minted}

\paragraph{} Then, locate the lines


\begin{minted}[linenos, frame=leftline]{php}
 <?php
    // ...
    if (!$this->server->verifyResourceRequest($request)) {
        $this->server->getResponse()->send();
        exit;
    }
    // ...
 ?>
\end{minted}

\paragraph{} And change them to the following code :

\begin{minted}[linenos, frame=leftline]{php}
 <?php
    // ...
    if (!$this->server->verifyResourceRequest($request)) {
        $response = $this->server->getResponse();
        if ($fromChatServer) {
            return $response;
        }
        else {
            $response->send();
            exit;
        }
    }
    // ...
 ?>
\end{minted}

\paragraph{Note:} If you are not using uJoin's source code, it is possible to 
change the OAuth authentication controller and method in the config.php file. 
Please make sure that the method takes a boolean as parameter to make the 
difference between calls from the chat server and other calls. Make sure to 
save the user model in the Config class with the key 'global.uJoinerAuth' in 
order for the server to be able to retrieve the user.

\subsection{Configuring the notifications}

\paragraph{} As part of this project I had to implement notifications for the 
chat. In order to comply to uJoin's API structure, I created two new models 
which need to be added to the source code. To do so, please execute the 
following command.

\begin{minted}{bash}
 $ cp $CHAT_ROOT/ChatConversationAction.php $LARAVEL_ROOT/app/models/
 $ cp $CHAT_ROOT/ChatNotification.php $LARAVEL_ROOT/app/models/
\end{minted}

\paragraph{} In the \$CHAT\_ROOT you will also find a README.md file which 
contains the remaining updates which need to be done in uJoin's source code in 
order for the notifications to be fully operational.

\paragraph{Note:} If you are not using uJoin's source code, it is possible to 
change the notification helper class in the config.php file. 
Please make sure that the methods described in the README.md file are present.

If you don't want to use notifications, please disable them in the config.php 
file by turning the enable\_notifs value to false.

\subsection{Starting the server}

\paragraph{} Finally, to start the chat server, you will need to execute the 
hereunder command. You can specify a different port than the default one which 
is 7778 by adding the optional argument --port.

\begin{minted}[linenos, frame=leftline]{php}
$ php artisan chat:server [--port=7778]
\end{minted}


\section{Installing the client}

\paragraph{} Please first copy the files located in \$CHAT\_ROOT/www\_src to \\
\$MOBILE\_ROOT/www\_src :

\begin{minted}{bash}
$ cp -r $CHAT_ROOT/www_src $MOBILE_ROOT/www_src
\end{minted}

\paragraph{} In order to be able to access the chat server, you will need to 
configure the server's host address and port in the file and change the 
corresponding values.

\paragraph{} File file \$MOBILE\_ROOT/www\_src/app/config/chatConfig.js
\begin{minted}{javascript}
.constant('server', {
    host: "127.0.0.1",
    port: 7778, 
    secured: false,
    testing: false
})
\end{minted}

\paragraph{} In order to be able to use the interface in uJoin's 
application please apply the changes detailed in the README.md file located in 
the \$MOBILE\_ROOT/www\_src folder.

\paragraph{Note :} The interface was built specifically for uJoin's client 
application source code, thus it will most likely not work if you use it in 
another context. The chat dependencies are described in the README.md file.

\section{Installing the Cordova plugin}

\paragraph{} Since the chat client makes use of the native Javascript WebSocket 
API, you need to install a cordova plugin in order to support them in the 
native Android application. You can install the plugin through the following 
command :

\begin{minted}{bash}
 $ cordova plugin add https://github.com/mkuklis/phonegap-websocket
\end{minted}

\paragraph{} This plugin does not add support for WebSockets on iOS 
applications. However, the WebSocket API is natively supported since iOS 4.2.
